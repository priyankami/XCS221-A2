\def\due{Sunday, November 29 at 11:59pm PT}
\def\assignmentnum{2 }
\def\assignmentname{Text Reconstruction}
\def\assignmenttitle{XCS221 Assignment \assignmentnum --- \assignmentname}



\documentclass{article}
\usepackage[top = 1.0in]{geometry}

\usepackage{graphicx}

\usepackage[utf8]{inputenc}
\usepackage{listings}
\usepackage[dvipsnames]{xcolor}

\definecolor{codegreen}{rgb}{0,0.6,0}
\definecolor{codegray}{rgb}{0.5,0.5,0.5}
\definecolor{codepurple}{rgb}{0.58,0,0.82}
\definecolor{backcolour}{rgb}{0.95,0.95,0.92}

\lstdefinestyle{mystyle}{
    backgroundcolor=\color{backcolour},   
    commentstyle=\color{codegreen},
    keywordstyle=\color{magenta},
    stringstyle=\color{codepurple},
    basicstyle=\ttfamily\footnotesize,
    breakatwhitespace=false,         
    breaklines=true,                 
    captionpos=b,                    
    keepspaces=true,                 
    numbersep=5pt,                  
    showspaces=false,                
    showstringspaces=false,
    showtabs=false,                  
    tabsize=2
}

\lstset{style=mystyle}

\usepackage[hyperfootnotes=false]{hyperref}
\usepackage{nccmath}
\usepackage{mathtools}
\usepackage{graphicx,caption}
\usepackage{enumitem}
\usepackage{epstopdf,subcaption}
\usepackage{psfrag}
\usepackage{amsmath,amssymb,epsf}
\usepackage{verbatim}
\usepackage{color,soul}
\usepackage{bbm}
\usepackage{listings}
\usepackage{setspace}
\usepackage{float}
\definecolor{Code}{rgb}{0,0,0}
\definecolor{Decorators}{rgb}{0.5,0.5,0.5}
\definecolor{Numbers}{rgb}{0.5,0,0}
\definecolor{MatchingBrackets}{rgb}{0.25,0.5,0.5}
\definecolor{Keywords}{rgb}{0,0,1}
\definecolor{self}{rgb}{0,0,0}
\definecolor{Strings}{rgb}{0,0.63,0}
\definecolor{Comments}{rgb}{0,0.63,1}
\definecolor{Backquotes}{rgb}{0,0,0}
\definecolor{Classname}{rgb}{0,0,0}
\definecolor{FunctionName}{rgb}{0,0,0}
\definecolor{Operators}{rgb}{0,0,0}
\definecolor{Background}{rgb}{0.98,0.98,0.98}
\lstdefinelanguage{Python}{
    numbers=left,
    numberstyle=\footnotesize,
    numbersep=1em,
    xleftmargin=1em,
    framextopmargin=2em,
    framexbottommargin=2em,
    showspaces=false,
    showtabs=false,
    showstringspaces=false,
    frame=l,
    tabsize=4,
    % Basic
    basicstyle=\ttfamily\footnotesize\setstretch{1},
    backgroundcolor=\color{Background},
    % Comments
    commentstyle=\color{Comments}\slshape,
    % Strings
    stringstyle=\color{Strings},
    morecomment=[s][\color{Strings}]{"""}{"""},
    morecomment=[s][\color{Strings}]{'''}{'''},
    % keywords
    morekeywords={import,from,class,def,for,while,if,is,in,elif,else,not,and,or
    ,print,break,continue,return,True,False,None,access,as,,del,except,exec
    ,finally,global,import,lambda,pass,print,raise,try,assert},
    keywordstyle={\color{Keywords}\bfseries},
    % additional keywords
    morekeywords={[2]@invariant},
    keywordstyle={[2]\color{Decorators}\slshape},
    emph={self},
    emphstyle={\color{self}\slshape},
%
}
\lstMakeShortInline|

\pagestyle{empty} \addtolength{\textwidth}{1.0in}
\addtolength{\textheight}{0.5in}
\addtolength{\oddsidemargin}{-0.5in}
\addtolength{\evensidemargin}{-0.5in}
\newcommand{\ruleskip}{\bigskip\hrule\bigskip}
\newcommand{\nodify}[1]{{\sc #1}}
\newenvironment{answer}{\sf \begingroup\color{ForestGreen}}{\endgroup}%

\setlist[itemize]{itemsep=2pt, topsep=0pt}
\setlist[enumerate]{itemsep=6pt, topsep=0pt}

\setlength{\parindent}{0pt}
\setlength{\parskip}{4pt}
\setlist[enumerate]{parsep=4pt}
\setlength{\unitlength}{1cm}

\renewcommand{\Re}{{\mathbb R}}
\newcommand{\R}{\mathbb{R}}
\newcommand{\what}[1]{\widehat{#1}}

\renewcommand{\comment}[1]{}
\newcommand{\mc}[1]{\mathcal{#1}}
\newcommand{\half}{\frac{1}{2}}

\def\KL{D_{KL}}
\def\xsi{x^{(i)}}
\def\ysi{y^{(i)}}
\def\zsi{z^{(i)}}
\def\E{\mathbb{E}}
\def\calN{\mathcal{N}}
\def\calD{\mathcal{D}}
\def\slack{\url{http://xcs221-scpd.slack.com/}}
\def\zipscriptalt{\texttt{python zip\_submission.py}}
\DeclarePairedDelimiter\abs{\lvert}{\rvert}%

\usepackage{bbding}
\usepackage{pifont}
\usepackage{wasysym}
\usepackage{amssymb}

% DO NOT MAKE EDITS TO THIS FILE
% POINT VALUES ARE ASSIGNED IN src/points.json
\usepackage{xstring}
\newcommand{\points}[1]{
  \IfEqCase{#1}{
    {1a}{\textbf{[1 point (Written)]}}
    {1b}{\textbf{[7 points (Coding)]}}
    {2a}{\textbf{[1 point (Written)]}}
    {2b}{\textbf{[10 points (Coding)]}}
    {3a}{\textbf{[1 point (Written)]}}
    {3b}{\textbf{[14 points (Coding)]}}
    {3c}{\textbf{[3 points (Written)]}}
    {3d}{\textbf{[1 point (Written)]}}
  }[\PackageError{points}{Cannot assign point value to unknown question: #1}{}]
}
% DO NOT MAKE EDITS TO THIS FILE
% POINT VALUES ARE ASSIGNED IN src/points.json


\begin{document}

\pagestyle{myheadings} \markboth{}{\assignmenttitle}

{\huge\noindent \assignmenttitle}

\ruleskip

{\bf Due {\due }.}

\medskip

\item {\bf Vowel Insertion}

Now you are given a sequence of English words with their vowels missing (A, E,
I, O, and U; never Y).  Your task is to place vowels back into these words in a
way that maximizes sentence fluency (i.e., that minimizes sentence cost).  For
this task, you will use a bigram cost function.

You are also given a mapping |possibleFills| that maps any vowel-free word to a
set of possible reconstructions (complete words).\footnote{This mapping was also
obtained by reading Tolstoy and Shakespeare and removing vowels.} For example,
|possibleFills(`fg')| returns |set([`fugue', `fog'])|.

\begin{enumerate}

  \item \points{2a}
Consider the following greedy-algorithm: from left to right, repeatedly pick the
immediate-best vowel insertion for the current vowel-free word, given the
insertion that was chosen for the previous vowel-free word. This algorithm does
{\em not} take into account future insertions beyond the current word.

Show, as in problem 1, that this greedy algorithm is suboptimal, by providing a
realistic counter-example using English text. Make any assumptions you'd like
about |possibleFills| and the bigram cost function, but bigram costs must remain
positive.


  \begin{answer}
  % ### START CODE HERE ###
  % ### END CODE HERE ###
\end{answer}


  \item \points{2b}
Implement an algorithm that finds optimal vowel insertions.  Use the UCS
subroutines.

When you've completed your implementation, the function |insertVowels| should
return the reconstructed word sequence as a string with space delimiters, i.e.
|` '.join(filledWords)|. Assume that you have a list of strings as the input,
i.e. the sentence has already been split into words for you. Note that the empty
string is a valid element of the list.

The argument |queryWords| is the input sequence of vowel-free words.  Note that
the empty string is a valid such word.  The argument |bigramCost| is a function
that takes two strings representing two sequential words and provides their
bigram score.  The special out-of-vocabulary beginning-of-sentence word
|-BEGIN-| is given by |wordsegUtil.SENTENCE_BEGIN|.  The argument
|possibleFills| is a function that takes a word as a string and returns a |set|
of reconstructions.

Since we use a limited corpus, some seemingly obvious strings may have no
filling, such as |chclt -> {}|, where |chocolate| is actually a valid filling.
Don't worry about these cases.

{\em Note: If some vowel-free word $w$ has no reconstructions according to
|possibleFills|, your implementation should consider $w$ itself as the sole
possible reconstruction.

Use the |ins| command in the program console to try your implementation.  For
example:

\begin{lstlisting}
>> ins thts m n th crnr

Query (ins): thts m n th crnr

thats me in the corner
\end{lstlisting}

The console strips away any vowels you do insert, so you can actually type in
plain English and the vowel-free query will be issued to your program.  This
also means that you can use a single vowel letter as a means to place an empty
string in the sequence.

For example:
\begin{lstlisting}
>> ins its a beautiful day in the neighborhood

Query (ins): ts  btfl dy n th nghbrhd

its a beautiful day in the neighborhood
\end{lstlisting}
}


\end{enumerate}


\ruleskip

\clearpage

In this homework, we consider two tasks: {\em word segmentation} and {\em vowel
insertion}.

Word segmentation often comes up when processing many non-English languages, in
which words might not be flanked by spaces on either end, such as written
Chinese or long compound German words.\footnote{In German,
``Windschutzscheibenwischer'' is ``windshield wiper''. Broken into parts:
``wind'' \textrightarrow ``wind''; ``schutz'' \textrightarrow ``block/
protection''; ``scheiben'' \textrightarrow ``panes''; ``wischer''
\textrightarrow ``wiper''.} Vowel insertion is relevant for languages like
Arabic or Hebrew, where modern script eschews notations for vowel sounds and the
human reader infers them from context.\footnote{See
\url{https://en.wikipedia.org/wiki/Abjad}.} More generally, this is an instance
of a reconstruction problem with a lossy encoding and some context.

We already know how to optimally solve any particular search problem with graph
search algorithms such as uniform cost search or A*.  Our goal here is modeling
--- that is, converting real-world tasks into state-space search problems.

{\bf Setup: $n$-gram language models and uniform-cost search}

Our algorithm will base its segmentation and insertion decisions on the cost of
processed text according to a {\em language model}. A language model is some
function of the processed text that captures its fluency.

A very common language model in NLP is an $n$-gram sequence model. This is a
function that, given $n$ consecutive words, provides a cost based on the
negative log likelihood that the $n$-th word appears just after the first $n-1$
words.\footnote{This model works under the assumption that text roughly
satisfies the \href{https://en.wikipedia.org/wiki/Markov_property} Markov
 property.}

The cost will always be positive, and lower costs indicate better fluency.
\footnote{Modulo edge cases, the $n$-gram model score in this assignment is
given by $\ell(w_1, \ldots, w_n) = -\log(p(w_n \mid w_1, \ldots, w_{n-1}))$.
Here, $p(\cdot)$ is an estimate of the conditional probability distribution over
words given the sequence of previous $n-1$ words.  This estimate is gathered
from frequency counts taken by reading Leo Tolstoy's {\em War and Peace} and
William Shakespeare's {\em Romeo and Juliet}.}

As a simple example: In a case where $n=2$ and $c$ is our $n$-gram cost
function, $c(${\sf big}, {\sf fish}$)$ would be low, but $c(${\sf fish},
{\sf fish}$)$ would be fairly high.

Furthermore, these costs are additive: For a unigram model $u$ ($n = 1$), the
cost assigned to $[w_1, w_2, w_3, w_4]$ is
\[
u(w_1) + u(w_2) + u(w_3) + u(w_4).
\]

Similarly, for a bigram model $b$ ($n = 2$), the cost is
\[
b(w_0, w_1) +
b(w_1, w_2) +
b(w_2, w_3) +
b(w_3, w_4),
\]

where $w_0$ is {\tt -BEGIN-}, a special token that denotes the beginning of the
sentence.

We have estimated $u$ and $b$ based on the statistics of $n$-grams in text. Note
that any words not in the corpus are automatically assigned a high cost, so you
do not have to worry about that part.

A note on low-level efficiency and expectations: This assignment was designed
considering input sequences of length no greater than roughly 200, where these
sequences can be sequences of characters or of list items, depending on the
task.  Of course, it's great if programs can tractably manage larger inputs, but
it's okay if such inputs can lead to inefficiency due to overwhelming state
space growth.

\clearpage

\begin{enumerate}[wide, labelindent=0pt]

  \item {\bf Vowel Insertion}

Now you are given a sequence of English words with their vowels missing (A, E,
I, O, and U; never Y).  Your task is to place vowels back into these words in a
way that maximizes sentence fluency (i.e., that minimizes sentence cost).  For
this task, you will use a bigram cost function.

You are also given a mapping |possibleFills| that maps any vowel-free word to a
set of possible reconstructions (complete words).\footnote{This mapping was also
obtained by reading Tolstoy and Shakespeare and removing vowels.} For example,
|possibleFills(`fg')| returns |set([`fugue', `fog'])|.

\begin{enumerate}

  \item \points{2a}
Consider the following greedy-algorithm: from left to right, repeatedly pick the
immediate-best vowel insertion for the current vowel-free word, given the
insertion that was chosen for the previous vowel-free word. This algorithm does
{\em not} take into account future insertions beyond the current word.

Show, as in problem 1, that this greedy algorithm is suboptimal, by providing a
realistic counter-example using English text. Make any assumptions you'd like
about |possibleFills| and the bigram cost function, but bigram costs must remain
positive.


  \begin{answer}
  % ### START CODE HERE ###
  % ### END CODE HERE ###
\end{answer}


  \item \points{2b}
Implement an algorithm that finds optimal vowel insertions.  Use the UCS
subroutines.

When you've completed your implementation, the function |insertVowels| should
return the reconstructed word sequence as a string with space delimiters, i.e.
|` '.join(filledWords)|. Assume that you have a list of strings as the input,
i.e. the sentence has already been split into words for you. Note that the empty
string is a valid element of the list.

The argument |queryWords| is the input sequence of vowel-free words.  Note that
the empty string is a valid such word.  The argument |bigramCost| is a function
that takes two strings representing two sequential words and provides their
bigram score.  The special out-of-vocabulary beginning-of-sentence word
|-BEGIN-| is given by |wordsegUtil.SENTENCE_BEGIN|.  The argument
|possibleFills| is a function that takes a word as a string and returns a |set|
of reconstructions.

Since we use a limited corpus, some seemingly obvious strings may have no
filling, such as |chclt -> {}|, where |chocolate| is actually a valid filling.
Don't worry about these cases.

{\em Note: If some vowel-free word $w$ has no reconstructions according to
|possibleFills|, your implementation should consider $w$ itself as the sole
possible reconstruction.

Use the |ins| command in the program console to try your implementation.  For
example:

\begin{lstlisting}
>> ins thts m n th crnr

Query (ins): thts m n th crnr

thats me in the corner
\end{lstlisting}

The console strips away any vowels you do insert, so you can actually type in
plain English and the vowel-free query will be issued to your program.  This
also means that you can use a single vowel letter as a means to place an empty
string in the sequence.

For example:
\begin{lstlisting}
>> ins its a beautiful day in the neighborhood

Query (ins): ts  btfl dy n th nghbrhd

its a beautiful day in the neighborhood
\end{lstlisting}
}


\end{enumerate}

  \clearpage

  \item {\bf Vowel Insertion}

Now you are given a sequence of English words with their vowels missing (A, E,
I, O, and U; never Y).  Your task is to place vowels back into these words in a
way that maximizes sentence fluency (i.e., that minimizes sentence cost).  For
this task, you will use a bigram cost function.

You are also given a mapping |possibleFills| that maps any vowel-free word to a
set of possible reconstructions (complete words).\footnote{This mapping was also
obtained by reading Tolstoy and Shakespeare and removing vowels.} For example,
|possibleFills(`fg')| returns |set([`fugue', `fog'])|.

\begin{enumerate}

  \item \points{2a}
Consider the following greedy-algorithm: from left to right, repeatedly pick the
immediate-best vowel insertion for the current vowel-free word, given the
insertion that was chosen for the previous vowel-free word. This algorithm does
{\em not} take into account future insertions beyond the current word.

Show, as in problem 1, that this greedy algorithm is suboptimal, by providing a
realistic counter-example using English text. Make any assumptions you'd like
about |possibleFills| and the bigram cost function, but bigram costs must remain
positive.


  \begin{answer}
  % ### START CODE HERE ###
  % ### END CODE HERE ###
\end{answer}


  \item \points{2b}
Implement an algorithm that finds optimal vowel insertions.  Use the UCS
subroutines.

When you've completed your implementation, the function |insertVowels| should
return the reconstructed word sequence as a string with space delimiters, i.e.
|` '.join(filledWords)|. Assume that you have a list of strings as the input,
i.e. the sentence has already been split into words for you. Note that the empty
string is a valid element of the list.

The argument |queryWords| is the input sequence of vowel-free words.  Note that
the empty string is a valid such word.  The argument |bigramCost| is a function
that takes two strings representing two sequential words and provides their
bigram score.  The special out-of-vocabulary beginning-of-sentence word
|-BEGIN-| is given by |wordsegUtil.SENTENCE_BEGIN|.  The argument
|possibleFills| is a function that takes a word as a string and returns a |set|
of reconstructions.

Since we use a limited corpus, some seemingly obvious strings may have no
filling, such as |chclt -> {}|, where |chocolate| is actually a valid filling.
Don't worry about these cases.

{\em Note: If some vowel-free word $w$ has no reconstructions according to
|possibleFills|, your implementation should consider $w$ itself as the sole
possible reconstruction.

Use the |ins| command in the program console to try your implementation.  For
example:

\begin{lstlisting}
>> ins thts m n th crnr

Query (ins): thts m n th crnr

thats me in the corner
\end{lstlisting}

The console strips away any vowels you do insert, so you can actually type in
plain English and the vowel-free query will be issued to your program.  This
also means that you can use a single vowel letter as a means to place an empty
string in the sequence.

For example:
\begin{lstlisting}
>> ins its a beautiful day in the neighborhood

Query (ins): ts  btfl dy n th nghbrhd

its a beautiful day in the neighborhood
\end{lstlisting}
}


\end{enumerate}

  \clearpage

  \item {\bf Vowel Insertion}

Now you are given a sequence of English words with their vowels missing (A, E,
I, O, and U; never Y).  Your task is to place vowels back into these words in a
way that maximizes sentence fluency (i.e., that minimizes sentence cost).  For
this task, you will use a bigram cost function.

You are also given a mapping |possibleFills| that maps any vowel-free word to a
set of possible reconstructions (complete words).\footnote{This mapping was also
obtained by reading Tolstoy and Shakespeare and removing vowels.} For example,
|possibleFills(`fg')| returns |set([`fugue', `fog'])|.

\begin{enumerate}

  \item \points{2a}
Consider the following greedy-algorithm: from left to right, repeatedly pick the
immediate-best vowel insertion for the current vowel-free word, given the
insertion that was chosen for the previous vowel-free word. This algorithm does
{\em not} take into account future insertions beyond the current word.

Show, as in problem 1, that this greedy algorithm is suboptimal, by providing a
realistic counter-example using English text. Make any assumptions you'd like
about |possibleFills| and the bigram cost function, but bigram costs must remain
positive.


  \begin{answer}
  % ### START CODE HERE ###
  % ### END CODE HERE ###
\end{answer}


  \item \points{2b}
Implement an algorithm that finds optimal vowel insertions.  Use the UCS
subroutines.

When you've completed your implementation, the function |insertVowels| should
return the reconstructed word sequence as a string with space delimiters, i.e.
|` '.join(filledWords)|. Assume that you have a list of strings as the input,
i.e. the sentence has already been split into words for you. Note that the empty
string is a valid element of the list.

The argument |queryWords| is the input sequence of vowel-free words.  Note that
the empty string is a valid such word.  The argument |bigramCost| is a function
that takes two strings representing two sequential words and provides their
bigram score.  The special out-of-vocabulary beginning-of-sentence word
|-BEGIN-| is given by |wordsegUtil.SENTENCE_BEGIN|.  The argument
|possibleFills| is a function that takes a word as a string and returns a |set|
of reconstructions.

Since we use a limited corpus, some seemingly obvious strings may have no
filling, such as |chclt -> {}|, where |chocolate| is actually a valid filling.
Don't worry about these cases.

{\em Note: If some vowel-free word $w$ has no reconstructions according to
|possibleFills|, your implementation should consider $w$ itself as the sole
possible reconstruction.

Use the |ins| command in the program console to try your implementation.  For
example:

\begin{lstlisting}
>> ins thts m n th crnr

Query (ins): thts m n th crnr

thats me in the corner
\end{lstlisting}

The console strips away any vowels you do insert, so you can actually type in
plain English and the vowel-free query will be issued to your program.  This
also means that you can use a single vowel letter as a means to place an empty
string in the sequence.

For example:
\begin{lstlisting}
>> ins its a beautiful day in the neighborhood

Query (ins): ts  btfl dy n th nghbrhd

its a beautiful day in the neighborhood
\end{lstlisting}
}


\end{enumerate}


\end{enumerate}

\end{document}
