\item \points{2b}
Implement an algorithm that finds optimal vowel insertions.  Use the UCS
subroutines.

When you've completed your implementation, the function |insertVowels| should
return the reconstructed word sequence as a string with space delimiters, i.e.
|' '.join(filledWords)|. Assume that you have a list of strings as the input,
i.e. the sentence has already been split into words for you. Note that the empty
string is a valid element of the list.

The argument |queryWords| is the input sequence of vowel-free words.  Note that
the empty string is a valid such word.  The argument |bigramCost| is a function
that takes two strings representing two sequential words and provides their
bigram score.  The special out-of-vocabulary beginning-of-sentence word
|-BEGIN-| is given by |wordsegUtil.SENTENCE_BEGIN|.  The argument
|possibleFills| is a function that takes a word as a string and returns a |set|
of reconstructions.

Since we use a limited corpus, some seemingly obvious strings may have no
filling, such as |chclt -> {}|, where |chocolate| is actually a valid filling.
Don't worry about these cases.

{\em Note: If some vowel-free word $w$ has no reconstructions according to
|possibleFills|, your implementation should consider $w$ itself as the sole
possible reconstruction.

Use the |ins| command in the program console to try your implementation.  For
example:

\begin{lstlisting}
>> ins thts m n th crnr

Query (ins): thts m n th crnr

thats me in the corner
\end{lstlisting}

The console strips away any vowels you do insert, so you can actually type in
plain English and the vowel-free query will be issued to your program.  This
also means that you can use a single vowel letter as a means to place an empty
string in the sequence.

For example:
\begin{lstlisting}
>> ins its a beautiful day in the neighborhood

Query (ins): ts  btfl dy n th nghbrhd

its a beautiful day in the neighborhood
\end{lstlisting}
}
